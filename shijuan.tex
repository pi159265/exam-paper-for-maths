% !Mode::"TeX:UTF-8" by pi159265
% 本模板是基于网上好几个模板修改而来,不过抱歉的是我具体记不清是那几个了
% 编码为UTF8, 需要用xelatex编辑, 否者会有错误.
\documentclass[twocolumn,landscape,UTF8]{article}
\usepackage{float,color,titlesec,graphicx,listings,makecell,fancyhdr,setspace}
\usepackage{amsmath,amsfonts,amsmath,amssymb,times,txfonts,ctex,zhnumber}
\setlength{\columnsep}{2cm}
\setlength{\marginparwidth}{10mm}
\usepackage{enumerate}% 编号
\usepackage[a3paper,top=2cm, right=0mm]{geometry}
\oddsidemargin=0.15cm   %奇数页页边距
\evensidemargin=0.15cm %偶数页页边距
\linespread{1.3}
\newcommand{\defen}{\zihao{3}\hspace*{11.1cm}\vspace{-19mm}
\renewcommand{\arraystretch}{1.1}
\begin{tabular}{|p{0.05\textwidth}|p{0.05\textwidth}|}
\hline
\centering 阅卷人& \\
\hline
\centering 得~~分 &  \\
	\hline
\end{tabular}}
\newsavebox{\zdx}
\newsavebox{\zdxr}
\newcommand{\putzdx}{\marginpar{\parbox{2cm}{\rotatebox[origin=c]{90}{\usebox{\zdx}}}}}%装订线-奇数页用
\newcommand{\putzdxr}{\marginpar{\parbox{2cm}{\rotatebox[origin=c]{270}{\usebox{\zdxr}}}}}%装订线-偶数页用
\renewcommand{\headrulewidth}{0pt}
\pagestyle{fancy}
\newcounter{tihao}
\newcommand{\timu}[1]{ \defen\\\begin{minipage}[b]{11.9cm}
{\protect\stepcounter{tihao}\zhnumber{\thetihao}\protect、}{#1}
\end{minipage}}%生成题目内容,会自动生成题目序号
\begin{document} 

% 生成奇数页密封线
\sbox{\zdx}
{\parbox{26cm}{\zihao{3}\centering\vspace{7mm}
	姓名~\underline{\makebox[44mm][c]{}}~ 学号\underline{\makebox[44mm][c]{}}~\CJKfamily{song} 专业\underline{\makebox[44mm][c]{}}~\CJKfamily{song} 院系\underline{\makebox[44mm][c]{}} ~\\
%答题时学号
\dotfill{} 密\dotfill{}封\dotfill{}{\small 密封线内请不要答题.} \dotfill{}线{\dotfill}}}
%偶数页密封线	
	\sbox{\zdxr}
{\parbox{26cm}{\zihao{3}~\\\centering\vspace{-4mm}
	姓名~\underline{\makebox[44mm][c]{}}~ 学号\underline{\makebox[44mm][c]{}}~\CJKfamily{song} 专业\underline{\makebox[44mm][c]{}}~\CJKfamily{song} 院系\underline{\makebox[44mm][c]{}} ~\\
	\vspace{2mm}
%答题时学号
\dotfill{} 密\dotfill{}封\dotfill{}{\small 密封线内请不要答题.} \dotfill{}线{\dotfill}}}
\begin{spacing}{1.25}
	\begin{center}
\begin{LARGE}
宇宙学院{2219-- 2220}学年第\,{1}\,学期\\
{星球大战课程期中考试}\,试卷\\
\end{LARGE}
(闭卷笔试\ \ 120 分钟)\\
	\vspace{0.5cm}
	\setlength{\tabcolsep}{6mm}
	\renewcommand\arraystretch{1.7} 
%%%%%%%%%%%%%%%%%%%%%%%%%%%以下为题目个数, 可以自行修改.	
\begin{tabular}{|c|c|c|c|c|c|c|c|c|c|}
	\hline
   题~号 &一 &二 &三 &四 &五 &六 &七 &八 & 总~分   \\
	\hline
   分~数 &   &  &  &   &  &  &  &  & \\
	 \hline
  阅卷人 &   &  &  &   &  &  &  &  &  \\
  \hline
\end{tabular}
\end{center}
\end{spacing}
\vspace{-0.5cm}
\setlength{\marginparsep}{1.5cm}
%%%%%%%%%%%%%%%%%%%%%%%%%%%%%%%%%%%%%%%%%%%%%%%%%%%%%5
%%%%%%%%%%%%%%%%%%%%%%%%%%%%%%%%%%%%%%%%%%%%%%%%%%%%%
%%%%%%%%%%%%%%%%%%%%%%%%%%%%%%%%%%%%%%%%%%%%%%%%%%%%%%%%%%

% 以下为题目输入位置, 以上为模板.		
\putzdx %%装订线--仅需添在奇页数, 如果试卷为单面打印, 每页均要添加. 打印时, 短边翻转.
\vspace{6mm}

\timu{(20分)~填空题

\begin{enumerate}\setcounter{enumi}{0}
			\item 我能吞下玻璃而不伤身体.
			\item 我能吞下玻璃而不伤身体.
			\item 我能吞下玻璃而不伤身体.
			\item 我能吞下玻璃而不伤身体.
\end{enumerate}}


\vspace{2mm}

\timu{(10 分)~我能吞下玻璃而不伤身体 }
 




\newpage 

 
 
\timu{(10 分)~我能吞下玻璃而不伤身体}



\vspace{8cm}

\timu{(15 分)~我能吞下玻璃而不伤身体}


\newpage  % 注意这为第一页第二版面

\timu{(10分)
我能吞下玻璃而不伤身体}

\vspace{8cm}


\timu{(10分)我能吞下玻璃而不伤身体}

\newpage

\putzdxr % 偶数页的装订线。 如果试卷为单面,不需要使用该命令.

 
 
 
\timu{(10分)~我能吞下玻璃而不伤身体.}
\vspace*{8.5cm}



 

\timu{(15分) 我能吞下玻璃而不伤身体. }

\end{document}

